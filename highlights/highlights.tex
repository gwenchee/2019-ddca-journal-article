\documentclass[review]{elsarticle}

\usepackage{lineno}
\usepackage{xspace}
\modulolinenumbers[5]

\journal{Advances in Engineering Software}

%% `Elsevier LaTeX' style
\bibliographystyle{elsarticle-num}
%%%%%%%%%%%%%%%%%%%%%%%

%%%% packages and definitions (optional)
\usepackage{placeins}
\usepackage{booktabs} % nice rules (thick lines) for tables
\usepackage{microtype} % improves typography for PDF
\usepackage{hhline}
\usepackage{amsmath}

%\usepackage[demo]{graphicx}
%\usepackage{caption}
%\usepackage{subcaption}

\usepackage{booktabs}
\usepackage{threeparttable, tablefootnote}

\usepackage{tabularx}
\newcolumntype{b}{>{\hsize=1.0\hsize}X}
\newcolumntype{s}{>{\hsize=.5\hsize}X}
\newcolumntype{m}{>{\hsize=.75\hsize}X}
\newcolumntype{x}{>{\hsize=.25\hsize}X}

\graphicspath{ {figures/} }

% tikz %
\usepackage{tikz}
\usetikzlibrary{positioning, arrows, decorations, shapes}

\usetikzlibrary{shapes.geometric,arrows}
\tikzstyle{process} = [rectangle, rounded corners, minimum width=3cm, minimum height=1cm,text centered, draw=black, fill=blue!30]
\tikzstyle{object} = [ellipse, rounded corners, minimum width=3cm, minimum height=1cm,text centered, draw=black, fill=green!30]
\tikzstyle{arrow} = [thick,->,>=stealth]

% hyperref %
\usepackage[hidelinks]{hyperref}

\begin{document}
\begin{frontmatter}
\title{Demand Driven Deployment Capabilities in Cyclus, a Fuel Cycle Simulator}
\end{frontmatter}
\section*{Highlights}
\begin{itemize}
       \item We developed the capability in Cyclus, a nuclear fuel cycle
       simulator, to automatically deploy fuel cycle facilities to 
       create a supply chain to meet user-defined power demand. 
       \item This new capability , d3ploy, successfully deployed fuel cycle facilities 
       in multiple transition scenarios from the current light water reactor 
       fleet to a closed fuel cycle with continuous recycling 
       in fast and thermal reactors.
       \item We conclude that using d3ploy to set up transition scenarios
       is more efficient than previous efforts that required a user to manually 
       calculate and use trial and error to set up the deployment scheme for the 
       supporting fuel cycle facilities. By automating
       this process, when the user varies input parameters in the simulation, d3ploy
       automatically adjusts the deployment scheme to meet the new constraints.
\end{itemize}
\end{document}
