\begin{abstract}
The present United States' nuclear fuel cycle faces challenges that hinder 
the expansion of nuclear energy technology. 
The U.S. Department of Energy identified four nuclear fuel cycle 
options, which make nuclear energy technology
more desirable. 
Successfully analyzing the transitions from the current 
fuel cycle to these promising fuel cycles requires a nuclear 
fuel cycle simulator that can predictively and automatically 
deploy fuel cycle facilities to meet user-defined power demand. 
This work developed the \deploy capability in \Cyclus, a nuclear fuel cycle
simulator, to automatically deploy fuel cycle facilities to meet 
user-defined power demand. 
User-controlled capabilities such as supply buffers, 
facility preferences, prediction algorithms, and installed capacity 
deployment were introduced to give users tools to minimize power 
undersupply in a transition scenario simulation. 
We demonstrate \deploy's capability to predict future commodities' 
supply and demand, and automatically deploy fuel 
cycle facilities to meet the predicted demand. 
We use \deploy to set up transition scenarios for promising 
nuclear fuel cycle options. 
\end{abstract}


